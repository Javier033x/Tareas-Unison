\documentclass[a4paper, 12pt]{article}

% -- Language --
\usepackage[spanish]{babel}
\usepackage[utf8]{inputenc}
\usepackage{enumitem}



% ----- Fonts -----
% -- Color --
\usepackage{xcolor}
\definecolor{azul}{RGB}{51,102,204}

% -- Page Margin --
\usepackage[margin=0.75in]{geometry}

% -- Espaciados --
\newcommand{\Pspace}{0.5cm}
\newcommand{\Aspace}{0.2cm}

\title
{
  Arquitectura de Computadoras \\
  2026 - 1
}

\begin{document}

\maketitle

\begin{center}
    \begin{tabular}{r|l}
        \textbf{Expediente} & \textbf{Nombre} \\ \hline
        223210350 & Amaya Soria Angel Alberto \\
        223216745 & Flores Salazar Luis Angel \\
        223215039 & Miranda Sanchez Javier Leonardo \\
        223203899 & Tostado Cortes Dante Alejandro
    \end{tabular}
\end{center}

\rule{\linewidth}{0.3mm}



% ---------- Tarea 1 ----------
\vspace{0.3cm}

\begin{center}
    { \LARGE Tarea 1}
\end{center}

\begin{center}
    \textbf{Nota:} En los ejercicios siguientes no escribir solo el resultado, se califica procedimiento y resultado.
\end{center}

\begin{enumerate}
    % - Problema 1
    \item Convierta los siguientes números a las bases indicadas.
    
    \begin{enumerate}[a)]
        \item Convertir el entero -72 de base 10 a binario con 8 bits.
        % Respuesta:
        \vspace{\Aspace} \par
        { \color{azul} 
        Conversion del valor absoluto 72: \\ \\
        \begin{tabular}{|c|c|c|}
        \hline
        Division & Cociente & Residuo \\
        $\frac{72}{2}$ & 36 & 0 \\
        $\frac{36}{2}$ & 18 & 0 \\
        $\frac{18}{2}$ & 9 & 0 \\
        $\frac{9}{2}$ & 4 & 1 \\
        $\frac{4}{2}$ & 2 & 0 \\
        $\frac{2}{2}$ & 1 & 0 \\
        $\frac{1}{2}$ & 0 & 1\\
        \hline
        \end{tabular} \\ \\

        $72_{10} = 01001000_2$ \\
        Intercambiamos ceros por unos y viceversa. \\
        Numero Invertido: $10110111$ \\ 
        Sumamos 1 al valor invertido: $10111000$ \\
        $-72_{10} =10111000_2 $
        
        }
        \item Convertir el entero 11001110 de base 2 con 8 bits a base 10. 
        % Respuesta:
        \vspace{\Aspace} \par
        { \color{azul} 
            $x = (1\cdot2^7) + (1\cdot2^6) + (0\cdot2^5) + (0\cdot2^4) + (1\cdot2^3) +(1\cdot2^2) + (1\cdot2^1) + (0\cdot2^0)$ \\
            $x = (1\cdot128) + (1\cdot64) + (0\cdot32) + (0\cdot16) + (1\cdot8) +(1\cdot4) + (1\cdot2) + (0\cdot1)$ \\
            $x = 128 + 64 + 8 + 4 + 2 = 206$
        }
        \item Convertir el número 1010010.101 de base 2 a base 16.
        % Respuesta:
        \vspace{\Aspace} \par
        { \color{azul} 
        Agregar los ceros necesarios: \\
        $01010010{.}1010$ \\
        Dividir en bloques: \\
        $0101-0010{.}-1010$ \\
        Convertir cada bloque de 4 digitos a hexadecimal: \\
        $52.A_{16}$
        
        }
        \item Convertir el número 48.D de base 16 a base 2. 
        % Respuesta:
        \vspace{\Aspace} \par
        { \color{azul} 
        $4 = 0100$ \\
        $8=1000$ \\
        $D=1101$ \\
        Juntamos todo: \\
        $01001000{.}1101_2$
        
        }
        \item Mostrar cómo se representa el número de punto flotante -121.47 
        (base 10) en el standard IEEE-754 con precisión sencilla. 
        % Respuesta:
        \vspace{\Aspace} \par
        { \color{azul} 
        El número es negativo, por lo tanto: \\
        $S = 1$ 
        
        Conversión del valor absoluto $121{.}47_{10}$ a binario:
        
        Parte entera: 
        $121_{10} = 1111001_2$
        
        Parte fraccionaria: \\
        $0{.}47 = \frac{47}{100}$ 
        
        Multiplicaciones sucesivas por 2: \\
        
        \begin{tabular}{|c|c|c|}
        \hline
        Paso & Resultado & Bit \\
        \hline
        $0{.}47 \times 2 = 0{.}94$ & 0.94 & 0 \\
        $0{.}94 \times 2 = 1{.}88$ & 0.88 & 1 \\
        $0{.}88 \times 2 = 1{.}76$ & 0.76 & 1 \\
        $0{.}76 \times 2 = 1{.}52$ & 0.52 & 1 \\
        $0{.}52 \times 2 = 1{.}04$ & 0.04 & 1 \\
        $0{.}04 \times 2 = 0{.}08$ & 0.08 & 0 \\
        $0{.}08 \times 2 = 0{.}16$ & 0.16 & 0 \\
        $0{.}16 \times 2 = 0{.}32$ & 0.32 & 0 \\
        $0{.}32 \times 2 = 0{.}64$ & 0.64 & 0 \\
        $0{.}64 \times 2 = 1{.}28$ & 0.28 & 1 \\
        $0{.}28 \times 2 = 0{.}56$ & 0.56 & 0 \\
        $0{.}56 \times 2 = 1{.}12$ & 0.12 & 1 \\
        $0{.}12 \times 2 = 0{.}24$ & 0.24 & 0 \\
        $0{.}24 \times 2 = 0{.}48$ & 0.48 & 0 \\
        $0{.}48 \times 2 = 0{.}96$ & 0.96 & 0 \\
        $0{.}96 \times 2 = 1{.}92$ & 0.92 & 1 \\
        $0{.}92 \times 2 = 1{.}84$ & 0.84 & 1 \\
        \hline
        \end{tabular} \\ 
        
        Parte fraccionaria en binario: \\
        $0{.}47_{10} = 0{.}01111000010100011_2$ 
        
        Número completo en binario: \\
        $121{.}47_{10} = 1111001{.}01111000010100011_2$ \\ \\
        
        Normalización: \\
        $1{.}11100101111000010100011_2 \times 2^6$ 
        
        Cálculo del exponente: \\
        $E = 6 + 127 = 133_{10} = 10000101_2$ \\ \\
        Representación IEEE-754 en precisión sencilla: \\
        
        \[
        \boxed{
        1\;\;10000101\;\;11100101111000010100011
        }
        \]
        }
    \end{enumerate}

    \newpage
    % - Problema 2
    \item Considerar dos implementaciones del mismo ISA. Las instrucciones se dividen 
    en 4 clases: A, B, C y D. P1 tiene un reloj de 2 GHz y CPIs de 1, 2, 3 y 3. P2 
    tiene un reloj de 3 GHz y CPIs de 2, 2, 4 y 4. Suponer que un programa hace 
    siete mil millones de instrucciones con la siguiente mezcla: 10\% de A, 20\% de 
    B, 50\% de C y 20\% de D. Responder las siguientes preguntas: 

    \begin{enumerate}[(a)]
        \item ¿Cuál implementación es más rápida? 
        % Respuesta:
        \vspace{\Aspace} \par
        { \color{azul} 
        Instrucciones por clase: \\
        $A = 10\% = 7 \times10^{9} \cdot 0{.}1 = 7\times10^{8}$ \\
        $B = 20\% = 7 \times10^{9} \cdot 0{.}2 = 14\times10^{8}$ \\
        $C = 50\% = 7 \times10^{9} \cdot 0{.}5 = 35\times10^{8}$ \\
        $D = 20\% = 7 \times10^{9} \cdot 0{.}2 = 14\times10^{8}$ \\ \\ 
        $P1:$ \\ 
        $CPIs$ 
        \begin{tabular}{|c|c|c|c|}
        \hline
        $A=1$ & $B=2$ & $C=3$ & $D=3$  \\
        \hline
        \end{tabular} \\
        $CiclosTotalesP1 = (1 \cdot 7 \times10^{8})+(2 \cdot 14 \times 10^{8})+ (3 \cdot 35 \times 10^{8})+(3 \cdot 14 \times 10^{8}) = 18{.}2\times10^{9}$ \\
        $Tiempo1=\frac{18{.}2\times10^{9}}{2\times10^{9}} = 9{.}1 $ Segundos
        \\ \\
        $P2:$ \\ 
        $CPIs$ 
        \begin{tabular}{|c|c|c|c|}
        \hline
        $A=2$ & $B=2$ & $C=4$ & $D=4$  \\
        \hline
        \end{tabular} \\
        $CiclosTotalesP2 = (2 \cdot 7 \times10^{8})+(2 \cdot 14 \times 10^{8})+ (4 \cdot 35 \times 10^{8})+(4 \cdot 14 \times 10^{8}) = 23{.}8\times10^{9}$ \\
        $Tiempo1=\frac{23{.}8\times10^{9}}{3\times10^{9}} = 7{.}93 $ Segundos\\ \\
        Respuesta: La implementación P2 es más rápida
        }
        \item ¿Cuál es el CPI del programa en cada implementación? 
        % Respuesta:
        \vspace{\Aspace} \par
        { \color{azul} 
        Instrucciones por clase, $CPIs1, CPIs2:$ \\
        $A = 10\%, 1, 2$ \\
        $B = 20\%, 2, 2$ \\
        $C = 50\%, 3, 4$ \\
        $D = 20\%, 3, 4$ \\ \\
        $CPI1 = (0{.}1\cdot1) + (0{.}2\cdot2)+(0{.}5\cdot3)+(0{.}2\cdot3) = 2{.}6$  \\
        $CPI2 = (0{.}1\cdot2) + (0{.}2\cdot2)+(0{.}5\cdot4)+(0{.}2\cdot4) = 3{.}4$ \\
        }
        \item ¿Cuántos ciclos hace el programa en cada implementación? 
        % Respuesta:
        \vspace{\Aspace} \par
        { \color{azul} 
        Tomamos los ciclos totales del ejercicio 2A \\
        $CiclosTotalesP1 = 18{.}2\times10^{9}$ \\
        $CiclosTotalesP2 = 23{.}8\times10^{9}$
        }
        \item ¿Cuál es la velocidad pico de cada implementación? 
        % Respuesta:
        \vspace{\Aspace} \par
        { \color{azul} 
        $VelocidadPico = \frac{Frecuencia}{CPIMinimo}$ \\
        $VelocidadPico1 = \frac{2\times10^{9}}{1} = 2\times10^{9}$ instrucciones/s \\
        $VelocidadPico2 = \frac{3\times10^{9}}{2} = 1{.}5\times10^{9}$ instrucciones/s  
        }
    \end{enumerate}

    \newpage
    % - Problema 3
    \item Suponer que, para un programa, el compilador A genera un número de 
    instrucciones de 1.0E9 y un tiempo de ejecución de 1.1 s, mientras que el 
    compilador B genera un número de instrucciones de instrucciones de 1.2E9 y 
    un tiempo de ejecución de 1.5 s.

    \begin{enumerate}[(a)]
        \item Encontrar el CPI promedio para cada programa, dado que el 
        procesador tiene un ciclo de reloj de 1 ns. 
        % Respuesta:
        \vspace{\Aspace} \par
        { \color{azul} 
        Formula a usar: \\
        $T = I\cdot CPI \cdot T_{c}$ \\
        $CPI = \frac{T}{I \cdot T_{c}}$ \\ \\
        Valor de $T_{c} =1 ns=1\times10^{-9}$ \\
        \begin{itemize}
            \item Compilador A: \\
            Datos: \\
            $I = 1\times 10^{9}$ \\
            $T = 1{.}1 $ s \\
            Proceso: \\
            $CPI_{A} = \frac{1{.}1}{(1\times 10^{9})\cdot (1\times10^{-9})}$
            $CPI_A=\frac{1{.}1}{1}=1{.}1$  
            \item Compilador B: \\
            Datos: \\
            $I = 1{.}2\times 10^{9}$ \\
            $T = 1{.}5 $ s \\
            Proceso: \\
            $CPI_{B} = \frac{1{.}5}{(1{.}2\times 10^{9})\cdot (1\times10^{-9})}$ \\
            $CPI_B=\frac{1{.}5}{1{.}2}=1{.}25$  
            
        \end{itemize}
        
        
        
        }
        \item Suponer que los programas compilados se ejecutan en dos 
        procesadores diferentes. Si los tiempos de ejecución en ambos 
        procesadores son iguales, ¿Cuánto más rápido es el reloj del procesador 
        que ejecuta el código del compilador A en comparación con el reloj del 
        procesador que ejecuta el código del compilador B?
        % Respuesta:
        \vspace{\Aspace} \par
        { \color{azul} 
        $T_A = T_B$, pero en diferentes procesadores \\
        Formula a usar: \\
        $T = I \cdot CPI \cdot T_C$ \\
        $I_A \cdot CPI_A \cdot T_{CA} = I_B \cdot CPI_B \cdot T_{CB}$ \\
        $f = \frac{1}{T_C}$ \\
        $\frac{T_{CA}}{T_{CB}} = \frac{I_B \cdot CPI_B}{I_A \cdot CPI_A} = \frac{(1{.}2\times10^9)\cdot(1{.}25)}{(1\times10^9)\cdot(1{.}1)}$ \\
        $\frac{T_{CA}}{T_{CB}} = \frac{1{.}5}{1{.}1} = 1{.}36$ \\ 
        
        }
        \newpage
        \item Se desarrolla un nuevo compilador C que utiliza solo 6.0E8 
        instrucciones y tiene un CPI promedio de 1.1. ¿Cuál es el speedup al usar 
        este nuevo compilador en comparación con usar el compilador A o el 
        compilador B en el procesador original?
        % Respuesta:
        \vspace{\Aspace} \par
        { \color{azul} 
        Datos:
        $I= 6\times10^8$ \\
        $CPI = 1{.}1$ \\
        $T_A = 1{.}1$ s \\
        $T_B = 1{.}5$ s \\ \\
        Formulas a usar: \\
        $SpeedUP = \frac{T_A}{T_C}$ \\
        $T_C = I \cdot CPI \cdot T_c$ \\ \\
        Proceso: \\
        $T_C = (6\times10^8)\cdot(1{.}1)\cdot(1\times10^{-9})$ \\
        $T_C = 0{.}66$ s \\
        $SpeedUP_{A/C} = \frac{1{.}1}{0{.}66} = 1{.}67$ \\
        $SpeedUP_{B/C} = \frac{1{.}5}{0{.}66} = 2{.}27$
        }
    \end{enumerate}
    
    \newpage
    % - Problema 3
    \item Suponer que a una computadora se le cambia la tarjeta gráfica por otra y que un benchmark tardaba en correr 90 segundos usando la tarjeta original. Al correr el benchmark con la tarjeta nueva se detecta un speedup de 1.5. ¿Qué tan rápida es la nueva tarjeta gráfica si se sabe que el benchmark pasa el 40\% del tiempo ejecutando instrucciones gráficas?

    \begin{enumerate}[(a)]
        \item Plantear claramente la ecuación a resolver.
        % Respuesta:
        \vspace{\Aspace} \par
        { \color{azul} 
            Usar ley de Amdahl, el speedup total de un sistema está dado por:

            \[
            S_{total} = \frac{1}{(1 - f) + \frac{f}{S_{mejorado}}}
            \]
            
            donde:
            \begin{itemize}
                \item $S_{total}$ es el speedup total del programa,
                \item $f$ es la fracción del tiempo de ejecución que se ve mejorada,
                \item $S_{mejorado}$ es el speedup de la parte mejorada del sistema.
            \end{itemize}
            
            La parte que se mejora aqui es la de las instrucciones gráficas, por lo que:
            
            \[
            S_{total} = 1{.}5
            \]
            \[
            f = 0{.}40
            \]
            \[
            1 - f = 0{.}60
            \]
            
            Sustituyendo estos valores en la ecuación da:
            
            \[
            1{.}5 = \frac{1}{0{.}60 + \frac{0{.}40}{S_{GPU}}}
            \]
            
            Esa es la ecuación que represnta el speedup de la nueva tarjeta gráfica.}
        \item Resolver la ecuación correctamente.
        % Respuesta:
        \vspace{\Aspace} \par
        { \color{azul} 
            Despejando $S_{GPU}$:

            \[
            \frac{1}{1{.}5} = 0{.}60 + \frac{0{.}40}{S_{GPU}}
            \]
            
            \[
            0{.}6667 = 0{.}60 + \frac{0{.}40}{S_{GPU}}
            \]
            
            
            \[
            0{.}0667 = \frac{0{.}40}{S_{GPU}}
            \]
            
            
            \[
            S_{GPU} = \frac{0{.}40}{0{.}0667}
            \]
            
            \[
            S_{GPU} \approx 6
            \]
        }
    \end{enumerate}
    
\end{enumerate}
\end{document}
